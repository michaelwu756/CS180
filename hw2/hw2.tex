\documentclass[12pt]{article}
\usepackage{amsmath}
\begin{document}
\title{Computer Science 180, Homework 2}
\date{January 25th, 2018}
\author{Michael Wu\\UID: 404751542}
\maketitle

\section*{Chapter 1, Problem 5}

\paragraph{a)}

The Gale-Shapely algorithm always produces a perfect matching with no strong instability,
if it is modified so that a woman \(w\) rejects any man \(m^\prime\) who asks her to be engaged
if she is already engaged to a man \(m\) who she prefers equally to the man \(m^\prime\).
In full the algorithm is as follows

\begin{verbatim}
Initially all men m in M and women w in W are free
While there are free men m who haven't proposed to every woman
    Choose such a man m
    Let w be one of m's unasked and most preferred women
    If w is free then
        (m,w) become engaged
    Else (m',w) are already engaged
        If w prefers m' more than or equal to m then
            m remains free
        Else
            (m,w) become engaged
            m' becomes free
        Endif
    Endif
Endwhile
Return the set S of engaged pairs
\end{verbatim}
This algorithm must generate a matching because the algorithm terminates after no men exist
who are free or have yet to propose to every woman. If no man is free, then everybody is matched.
If a man has proposed to every woman, then every woman should have been proposed to at least once.
This means that every woman is paired because they must remain paired after being proposed to once.
We prove that this results in a perfect matching with no strong instability by contradiction. Assume
that this algorithm produces a matching \(S\) with a strong instability. Thus there exists the pairs
\((m,w)\in S\) and \((m^\prime,w^\prime)\in S\) where \(m\) prefers \(w^\prime\) to \(w\) and where
\(w^\prime\) prefers \(m\) to \(m^\prime\). Then \(m\) has already asked \(w^\prime\) to be engaged
before asking \(w\) because he asks in order of decreasing preference. Afterwards, \(m^\prime\) has asked
\(w^\prime\) to be engaged, causing her to cancel the engagement with \(m\) because she prefers \(m^\prime\)
to \(m\). This is a contradiction because \(w^\prime\) prefers \(m\) to \(m^\prime\) in order for there to
be a strong instability. Thus no strong instability can occur.

\paragraph{b)}

Here is a set of two men and two women with preference lists where any perfect matching always results in a
weak instability. Man \(m_1\) prefers \(w_1\) to \(w_2\). Man \(m_2\) prefers \(w_1\) to \(w_2\). Woman
\(w_1\) prefers \(m_1\) and \(m_2\) equally. Woman \(w_2\) prefers \(m_1\) and \(m_2\) equally. There are only
two perfect matchings. The first is the perfect matching \((m_1,w_1), (m_2,w_2)\). A weak instability occurs because \(m_2\)
prefers \(w_1\) to \(w_2\) and \(w_1\) is indifferent between \(m_2\) and \(m_1\). The second is the perfect matching
\((m_1,w_2), (m_2,w_1)\). A weak instability occurs because \(m_1\) prefers \(w_1\) to \(w_2\) and \(w_1\) is
indifferent between \(m_1\) and \(m_2\).

\pagebreak

\section*{Chapter 1, Problem 8}

The following preference list allows for a switch that would improve the partner of a woman \(w_1\) who switched preferences.

\[
        \begin{array}{cccc}
                \text{person} & 1\text{st pref.} & 2\text{nd pref.} & 3\text{rd pref.} \\
                m_1 & w_1 & w_3 & w_2\\
                m_2 & w_1 & w_2 & w_3\\
                m_3 & w_3 & w_1 & w_2\\
                w_1 & m_3 & m_1 & m_2\\
                w_2 & m_2 & m_3 & m_1\\
                w_3 & m_1 & m_3 & m_2
        \end{array}
\]
This results in the stable matching \(S=\left\{(m_1,w_1),(m_2,w_2),(m_3,w_3)\right\}\). If \(w_1\) switched \(m_1\) and \(m_2\), this
would result in the stable matching \(S^\prime=\left\{(m_1,w_3),(m_2,w_2),(m_3,w_1)\right\}\). This is beneficial to \(w_1\).

\pagebreak

\section*{Chapter 2, Problem 4}

\begin{enumerate}
        \item \(g_1(n)=2^{\sqrt{\log n}}\)
        \item \(g_3(n)=n(\log n)^3\)
        \item \(g_4(n)=n^\frac{4}{3}\)
        \item \(g_5(n)=n^{\log n}\)
        \item \(g_2(n)=2^n\)
        \item \(g_7(n)=2^{n^2}\)
        \item \(g_6(n)=2^{2^n}\)
\end{enumerate}

\pagebreak

\section*{Chapter 2, Problem 5}

\paragraph{a)}

True. \(\log_2(n)\) is strictly increasing on \((0,\infty)\). As \(f(n)\) grows
large, \(f(n)<cg(n)\) for some constant \(c\). Thus
\[\log_2 f(n)<\log_2 cg(n)=\log_2 g(n) + \log_2(c)\]
We can ignore the constant value because it does not affect the growth rate at
all, and so \(\log_2 f(n)\) is \(O(\log_2 g(n))\).

\paragraph{b)}

False. \(f(n)=x\) and \(g(n)=\frac{x}{2}\). Then \(f(n)\) is \(O(g(n))\) but 
\(2^x\) is not \(O(2^\frac{x}{2})\).

\paragraph{c)}

True. We have that \(f(n)<cg(n)\) for some constant \(c\) as \(n\) grows large. Thus
\[f(n)^2<c^2g(n)^2\]
Because \(c^2\) is just a constant, we have that \(f(n)^2\) is \(O(g(n)^2)\).

\end{document}