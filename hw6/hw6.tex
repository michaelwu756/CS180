\documentclass[12pt]{article}
\usepackage{amsmath}
\usepackage{amssymb}
\begin{document}
\title{Computer Science 180, Homework 6}
\date{February 22nd, 2018}
\author{Michael Wu\\UID: 404751542}
\maketitle

\section*{Chapter 6, Problem 19}

First we set \(x^\prime=x^k\) for some \(k\) such that \(x^\prime\) is at least the length of \(s\). Similarly for \(y\)
let \(y^\prime=y^k\) for some \(k\) such that \(y^\prime\) is at least the length of \(s\). If \(s\) is an interleaving of
\(x\) and \(y\), denote \(s^\prime\) to be the subsequence of \(s\) that is a repetition of \(x\) and denote \(s^{\prime\prime}\)
to be the subsequence of \(s\) that is a repetition of \(y\). To solve this problem, we consider the subproblems \(\operatorname{OPT}(i,j)\) where
\[
        \operatorname{OPT}(i,j)=\begin{cases}
                \text{true} & \text{if }\exists s^\prime\text{ s.t. } |s^\prime|=i\text{ and }\exists s^{\prime\prime}\text{ s.t. }|s^{\prime\prime}|=j\\
                \text{false} & \text{otherwise}
        \end{cases}
\]
for a string \(t\) of length \(|t|=i+j\). Our base case is \(\operatorname{OPT}(0,0)=\text{true}\). Then for our given string \(s\), we consider the substring
\(s_1\) containing only the first character . This is an interleaving of \(x\) and \(y\) if and only if \(\operatorname{OPT}(1,0)=\text{true}\) or
\(\operatorname{OPT}(0,1)=\text{true}\). Let us access the \(i\)th character of a string \(a\) using the zero indexed notation \(a[i-1]\). \(\operatorname{OPT}(1,0)\)
and \(\operatorname{OPT}(0,1)\) are true only if \(s[0]=x^\prime[0]\) or \(s[0]=y^\prime[0]\), respectively. This tells us whether \(s_1\) is an interleaving of \(x\)
and \(y\). Then we can consider the substring \(s_2\) containing the first and second characters, and calculate \(\operatorname{OPT}(2,0)\),
\(\operatorname{OPT}(1,1)\), and \(\operatorname{OPT}(0,2)\) using the following recurrence
\[
        \operatorname{OPT}(i,j)=\begin{cases}
                \text{true} & \text{if }\operatorname{OPT}(i-1,j)=\text{true and }x^\prime[i-1]=s[i+j-1]\\
                \text{true} & \text{if }\operatorname{OPT}(i,j-1)=\text{true and }y^\prime[j-1]=s[i+j-1]\\
                \text{false} & \text{otherwise}
        \end{cases}
\]
which corresponds to assigning the next character in our string to either \(x^\prime\) or \(y^\prime\). We can continue to use this recurrence,
each time considering a string that increases in length by \(1\), until we calculate every possible \(\operatorname{OPT}(i,j)\) such that \(i+j\) is
equal to the length of our original string \(s\). If such an \(\operatorname{OPT}(i,j)=\text{true}\), then it is true that \(s\) is an interleaving
of \(x\) and \(y\). Otherwise \(s\) cannot be an interleaving of \(x\) and \(y\). In full our algorithm is as follows.

\begin{verbatim}
boolean checkInterleaving(String s, String x, String y){
   while(x.size()<s.size())
      x=x.concat(x);
   while(y.size()<s.size())
      y=y.concat(y);
   boolean[][] opt=new boolean[s.size()+1][s.size()1+1];
   opt[0][0]=true;
   for(int i=1;i<=s.size();i++)
      for(int j=0;j<=i;j++){
         opt[i-j][j]=false;
         if(i-j-1>0 && opt[i-j-1][j]==true
         && x.charAt(i-j-1)==s.charAt(i-1))
            opt[i-j][j]=true;
         if(j-1>0 && opt[i-j][j-1]==true
         && y.charAt(j-1)==s.charAt(i-1))
            opt[i-j][j]=true;
         if(i==s.size() && opt[i-j][j]==true)
            return true;
      }
   return false;
}
\end{verbatim}
This has a runtime of at most \(O(s^2)\), because we calculate a \(2\)-D array that has a height and width that grows linearly with \(s\). Its correctness
is a result of the recurrence relation given above.

\pagebreak

\section*{Chapter 5, Problem 2}

We simply modify the inversion algorithm by maintaining another pointer \(p\) that gives us how many significant inversions there are when merging.
More explicitly, when we place an element \(r\) from the right half \(R\) before the \(i\)th element of the left half \(L\), we set \(p=i\) and increment it by one until
we get the \(p=j\)th element \(l_j\) such that \(l_j>2r\). Then we know there are \(|L|-j+1\) significant inversions caused by this placement of \(r\) because \(L\)
is sorted. For subsequent placements of elements \(r\in R\), we only change \(p\) by incrementing it or moving forward to match the first unplaced element in \(L\). Thus
\(p\) traverses \(L\) once, which takes linear time. So our merge function still takes linear time, leading to an overall \(O(n \log n)\) runtime for our inversion algorithm.
In full this algorithm is as follows.
\begin{verbatim}
int mergeCountInv(List<Integer> S,int start,int end) {
   if (end-start<=1)
      return 0;
   int mid=(start+end)/2;
   int inversions=0;
   inversions+=mergeCountInv(S,start,mid);
   inversions+=mergeCountInv(S,mid,end);
   inversions+=merge(S,start,mid,end);
   return inversions;
}
int merge(List<Integer> S,int start,int mid,int end) {
   List<Integer> merged=new ArrayList<Integer>();
   int inversions=0;
   int indL=start;
   int indR=mid;
   int indInv=start;
   while (indL<mid || indR<end)
      if (indL==mid)
         merged.add(S.get(indR++));
      else if (indR==end)
         merged.add(S.get(indL++));
      else if (S.get(indL)<=S.get(indR)) {
         merged.add(S.get(indL++));
      else {
         merged.add(S.get(indR++));
         if(indInv<indL)
            indInv=indL;
         while(S.get(indInv)<=2*S.get(indR) && indInv<mid)
            indInv++;
         inversions+=mid-indInv;
      }
   for (int i=0;i<merged.size();i++)
      S.set(start+i, merged.get(i));
   return inversions;
}
\end{verbatim}
Here we merge sort our list \(S\) in place, and use the integer \texttt{indInv} as a pointer to track how many significant inversions there are when merging.

\pagebreak

\section*{Chapter 5, Problem 3}

\begin{verbatim}
(Card, int) checkFraud(List<Card> S) {
   if(S.length==1)
      return (S.get(0),1);
   List<Card> left=S.subList(0,S.size()/2);
   List<Card> right=S.subList(S.size()/2,S.size());
   Card leftCard, int leftSize=checkFraud(left);
   Card rightCard, int rightSize=checkFraud(right);
   if(leftCard==null && rightCard==null)
      return (null,0);
   int size=0;
   Card card=null;
   if(rightCard!=null) {
      int matches=0;
      for(int i=0;i<left.size();i++)
         if(sameAccount(rightCard,left.get(i)))
            matches++;
      if(matches+rightSize>S.size()/2) {
         size=matches+rightSize;
         card=rightCard;
      }
   }
   if(leftCard!=null) {
      int matches=0;
      for(int i=0;i<right.size();i++)
         if(sameAccount(rightCard,left.get(i)))
            matches++;
      if(matches+leftSize>S.size()/2) {
         size=matches+leftSize;
         card=leftCard;
      }
   }
   return (card, size)
}
\end{verbatim}

\pagebreak

\section*{Chapter 5, Problem 5}

\begin{verbatim}
List<Line> getVisibleLines(List<Line> lines) {
   lines=mergeSortBySlope(lines);
   int i=0;
   while(i<lines.size()-1)
      if(lines.get(i).m()==lines.get(i+1).m())
         if(lines.get(i).b()>=lines.get(i+1).b())
            lines.remove(i+1);
         else
            lines.remove(i);
      else
         i++;
   if(lines.size()<=2)
      return lines;
   List<Line> visible = new ArrayList<Line>();
   visible.add(lines.get(0));
   visible.add(lines.get(1));
   int numVisible=2;
   for(i=2;i<lines.size();i++) {
      Line prev=visible.get(numVisible-1);
      Line secondPrev=visible.get(numVisible-2);
      visible.add(lines.get(i));
      numVisible++;
      double prevX = (prev.b()-secondPrev.b())/(prev.m()-secondPrev.m());
      double currX = (lines.get(i).b()-secondPrev.b())/(lines.get(i).m()-secondPrev.m());
      while(currX<prevX) {
         visible.remove(prevX);
         numVisible—;
         if(numVisible==2)
            break;
         prev = visible.get(numVisible-2);
         secondPrev = visible.get(numVisible-3);
         prevX = (prev.b()-secondPrev.b())/(prev.m()-secondPrev.m());
         currX = (lines.get(i).b()-secondPrev.b())/(lines.get(i).m()-secondPrev.m());
      }
   }
   return visible;
}
\end{verbatim}

\pagebreak

\end{document}