\documentclass[12pt]{article}
\usepackage{amsmath}
\begin{document}
\title{Computer Science 180, Homework 1}
\date{January 18th, 2018}
\author{Michael Wu\\UID: 404751542}
\maketitle

\section*{Chapter 2, Problem 2}

Total number of operations: \(3600*10^{10}\) ops

\paragraph{a)}

\begin{align*}
        n^2&=3600*10^{10}\\
        n&=6000000
\end{align*}

\paragraph{b)}

\begin{align*}
        n^3&=3600*10^{10}\\
        n&\approx 33019
\end{align*}

\paragraph{c)}

\begin{align*}
        100n^2&=3600*10^{10}\\
        n^2&=3600*10^{8}\\
        n&=600000
\end{align*}

\paragraph{d)}

\begin{align*}
        n\log n&=3600*10^{10}\\
        n&\approx 1290951819848
\end{align*}

\paragraph{e)}

\begin{align*}
        2^n&=3600*10^{10}\\
        n&\approx 45
\end{align*}

\paragraph{f)}

\begin{align*}
        2^{2^n}&=3600*10^{10}\\
        n&\approx 5
\end{align*}

\section*{Chapter 2, Problem 3}

\begin{enumerate}
        \item \(f_2(n)=\sqrt{2n}\)
        \item \(f_3(n)=n+10\)
        \item \(f_6(n)=n^2 \log n\)
        \item \(f_1(n)=n^{2.5}\)
        \item \(f_4(n)=10^n\)
        \item \(f_5(n)=100^n\)
\end{enumerate}

\section*{Chapter 2, Problem 7}

Assume for simplification that each line is the maximum length \(c\) words. Then for \(x\) lines the total number of words \(n\) is
\begin{align*}
        n&=\sum_{i=1}^x ci\\
        n&=c\frac{x(x+1)}{2}\\
        n&=\frac{c}{2}(x^2+x)\\
\end{align*}
Solving this for \(x\) allows us to calculate the number of lines given a song with a total number of words \(n\)
\begin{align*}
        x^2+x&=\frac{2n}{c}\\
        x^2+x+\frac{1}{4}&=\frac{2n}{c}+\frac{1}{4}\\
        \left(x+\frac{1}{2}\right)^2&=\frac{2n}{c}+\frac{1}{4}\\
        x+\frac{1}{2}&=\sqrt{\frac{2n}{c}+\frac{1}{4}}\\
        x&=\frac{1}{2}\left(\sqrt{\frac{8n}{c}+1}-1\right)
\end{align*}
To encode the song of length \(n\) words, simply write down each unique line. This requires \(cx\) words. Thus our encoding as a function of \(n\)
has a length 
\[f(n)=\frac{c}{2}\left(\sqrt{\frac{8n}{c}+1}-1\right)\]
when \(\frac{n}{c}\) is an integer. If some lines are less than \(c\) words, then we should be able to encode the song in a fewer amount of words than
this because more words will be repeated. To make our \(f(n)\) account for this, and for the fact that \(\frac{n}{c}\) must be an integer, we change it to
\[f(n)=\frac{c}{2}\left(\sqrt{8\left\lceil\frac{n}{c}\right\rceil+1}-1\right)\]
This gives an upper bound for the number of words needed to encode a song of length \(n\) words. Thus by encoding the song by writing each unique line, our
encoding grows in \(O\left(\sqrt{n}\right)\) time.

\section*{Chapter 2, Problem 8}

\paragraph{a)}
For \(n\) rungs our strategy consists of dropping the first jar beginning at the lowest rung. Then continuously try each rung
\(\left\lceil(\sqrt{n})\right\rceil\) above the previous until the first jar breaks. Then starting from the highest rung from which the first jar did not break,
try moving up one rung at a time until you reach the highest safe rung, after which you will finish your testing. The first jar will break after at most
\(\left\lceil(\sqrt{n})\right\rceil\) tries, and the second jar will break after at most \(\left\lceil(\sqrt{n})\right\rceil\) tries as well.
Thus our solution requires at most \(f(n)=2\left\lceil(\sqrt{n})\right\rceil\) tries which is an \(O\left(\sqrt{n}\right)\) algorithm. This is better than a linear
\(O(n)\) solution.

\paragraph{b)}
For \(k>2\) jars available to break, drop the \(1\)st jar starting from the bottom of the ladder while moving up in intervals of \(\left\lceil n^\frac{k-1}{k}\right\rceil\)
rungs. When it breaks, move to the next jar. Continue by dropping the \(i\)th jar from the previous jar's highest rung reached before breaking, then moving up in intervals
of \(\left\lceil n^\frac{k-i}{k}\right\rceil\) rungs. Each jar can be dropped a maximum of \(\left\lceil n^\frac{1}{k}\right\rceil\) times and there are \(k\) jars, so
this algorithm in the worse case requires \(f_k(n)=k \left\lceil n^\frac{1}{k}\right\rceil\) steps. Thus this algorithm is \(O\left(n^\frac{1}{k}\right)\). This algorithm's
runtime satisfies the property that \(\lim_{n\to\infty} f_k(n)/f_{k-1}(n)=0\) for each \(k\).

\end{document}