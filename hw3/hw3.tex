\documentclass[12pt]{article}
\usepackage{amsmath}
\begin{document}
\title{Computer Science 180, Homework 3}
\date{February 1st, 2018}
\author{Michael Wu\\UID: 404751542}
\maketitle

\section*{Chapter 8, Problem 15}

We will show that the Nearby Electromagnetic Observation problem is NP-complete by first showing
that a solution can be checked in polynomial time, then by showing that the Vertex Cover problem reduces
to the Nearby Electromagnetic Observation problem.

To show that a solution can be checked in polynomial time, we show that we can verify whether a set of locations is
sufficient in polynomial time. Assume there are \(n\) frequencies and \(m\) locations.
Given a set of locations \(L^\prime\) with size \(k\leq m\) that form a sufficient set for the Nearby Electromagnetic Observation problem, we can
check whether or not \(L^\prime\) is correct by going through each of \(b\) interference sources \((F_i, L_i)\) for \(1\leq i\leq b\).
Begin with a set \(P\) of at most size \(nm\)\ containing all pairs \((f_j, l_k)\) of locations in \(L^\prime\) to all frequencies.
For each interference source, if a location is blocked from a frequency remove the corresponding pair from \(P\).
Then check that the number of unique frequencies in \(P\) is equal to \(j\). If this is true \(L^\prime\) is a solution
to the Nearby Electromagnetic Observation problem. This algorithm has the time complexity \(O(bnm)\), proving that the
Nearby Electromagnetic Observation problem can be verified in polynomial time. Thus it is \(NP\).

To show that Vertex Cover\(\leq_P\)Nearby Electromagnetic Observation, we begin by noting that the Nearby Electromagnetic Observation
problem is similar to finding the Vertex Cover of a bipartite graph, with one set of vertices \(F\) and another set of vertices \(L\). Thus we can reduce
the Vertex Cover problem to the Nearby Electromagnetic Observation problem by converting a graph \(G=(V,E)\) to a bipartite graph. We do this by splitting
each edge \(e\in E\) and inserting a vertex. Let this inserted vertex \(f\) be in the set \(F\). Each vertex \(f\) only connects to vertices in \(V\), thus
we have a bipartite graph. \(V\) corresponds to the locations in the Nearby Electromagnetic Observation problem and \(F\) corresponds to the frequencies.
Create a set of interference sources such that each frequency can only be received by two locations in \(V\). These two locations correspond to the edges in the original
graph \(G\). Thus for the Vertex Cover problem, we can construct a set of frequencies, locations, and interference sources that represent \(G\). Then solving
the Nearby Electromagnetic Observation problem gives a set of locations that are sufficient for the representation of \(G\). This set of locations also form
a vertex cover for \(G\), as each frequency will be accessed. The frequencies correspond to the edges in \(G\), thus we obtain a set of vertices that cover
each edge. This gives us a solution to the vertex cover problem. Thus
\[\text{Vertex Cover}\leq_P\text{Nearby Electromagnetic Observation}\]
which proves that the Nearby Electromagnetic Observation problem is NP-complete.

\pagebreak

\section*{Chapter 8, Problem 18}



\pagebreak

\section*{Chapter 8, Problem 22}

First we need to split up \(G=(V,E)\) into connected subgraphs. Do this as follows

\begin{verbatim}
initialize an empty set S of all subgraphs
for all v in V
    create a subgraph g
    add v to subgraph g
    remove v from V
    for all v_graph in subgraph g
        for all e in E
            if e connects v_graph with v_other
                add e to subgraph g
                add v_other to subgraph g
                remove e from E
                remove v_other from V
            endif
        endfor
    endfor
    add subgraph g to S
endfor
return S
\end{verbatim}

Because this algorithm contains three loops over elements of \(G\), it is \(O(G^3)\), which is polynomial time.

Then for each subgraph \(g\in S\), call \(A\) beginning with \(k=1\) on each subgraph and increasing \(k\) by one
each iteration until \(A\) returns ``no". Then we know the independent set \(g\) has a maximum independent set of size
\(k-1\). Summing the maximum independent set sizes of all subsets yields a result \(x\), the maximum independent set size
of \(G\). This allows us to solve the Independent Set Problem in polynomial time, as the number of calls to \(A\) is
\(O(G)\).

\pagebreak

\section*{Chapter 8, Problem 31}

\pagebreak

\section*{Chapter 8, Problem 36}

\end{document}