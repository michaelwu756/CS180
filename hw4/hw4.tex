\documentclass[12pt]{article}
\usepackage{amsmath}
\usepackage{amssymb}
\begin{document}
\title{Computer Science 180, Homework 4}
\date{February 8th, 2018}
\author{Michael Wu\\UID: 404751542}
\maketitle

\section*{Chapter 4, Problem 3}

Assume that there is some optimal packing of packages \(S\) such that we can ship all the packages using \(n\)
trucks. Then we can transform this packing into another optimal packing \(S^\prime\) produced by our greedy packing
algorithm. Let a truck that cannot hold the next package \(i\) in the mailing sequence, either because doing so would exceed the weight
limit \(W\) or because no more packages remain, be called a full truck. Our greedy packing algorithm generates a
sequence of full trucks. Now we will show that \(S\) can be transformed into a sequence of full trucks as well. If \(S\)
does not assign the first truck \(t_1\) enough packages such that it is a full truck, then we can remove the next package
\(i\) from the following truck \(t_2\) and place it in \(t_1\). This is a new optimal packing, as \(t_2\) now holds \(w_i\)
less weight, and \(t_1\) does not exceed the weight limit because it was not full prior to loading package \(i\). We continue moving packages
from \(t_2\) to \(t_1\) until \(t_1\) is full. The same number of trucks \(n\) are used, but our packing now starts with a full truck. So
any optimal packing \(S\) can be modified into another optimal packing that starts with a full truck. Looking at the remaining \(n-1\) trucks from
\(t_2\) to \(t_n\), we have another optimal packing of the remaining packages \(S-\{\text{packages in }t_1\}\), since \(S\) is originally an
optimal packing. We can make the first truck in this packing full as well. Repeating this procedure until we reach the last truck, we see that we
have created a packing \(S^\prime\) consisting entirely of full trucks that uses the same number \(n\) trucks as \(S\). Thus our greedy
algorithm, which generates \(S^\prime\), generates an optimal solution.

\pagebreak

\section*{Chapter 4, Problem 7}

An algorithm that generates a schedule with the shortest completion time is scheduling the shortest finishing time last.
This algorithm can be implemented in \(O(n \log n)\) time using merge sort. The psuedocode is as follows

\begin{verbatim}
Merge Sort list of jobs L by the finish time
Initialize empty schedule S
for i from 0 to (length of L)-1
    S[(length of L)-1-i]=L[i]
return S
\end{verbatim}
This is polynomial time, as the most expensive operation is the sorting. Otherwise, it only traverses through the list of jobs once.

Before we prove that this algorithm is correct, note that the shortest time on the supercomputer is \(\sum p_i\), because all
the jobs must be processed sequentially. Thus the completion time must be at least \(\sum p_i\). We want show that any optimal solution
\(S\) that has a minimum completion time \(t_a\) can contain the shortest finishing time job \(j_s\) last. If the shortest finishing time
job \(j_s\) is not last, then swapping it with the current last job \(j_l\) cannot increase the completion time. Note that \(t_a\)
must be at least as much as the preprocessing time plus the finishing time \(f_l\) of the last job, so \(t_a\geq\sum p_i + f_l\). Let
\(\sum p_i + f_l\) be called the completion time \(t_l\) of the job \(j_l\). If we swap \(j_s\) and \(j_l\), the completion time of \(j_s\)
will be \(\sum p_i + f_s\). We have that \(t_a\geq\sum p_i + f_l\geq\sum p_i + f_s\) because \(f_l>f_s\), so the total completion time is not increased
by positioning the shortest finishing time last. The total completion time is also not increased by moving the job \(j_l\) backwards, because it will complete in \(t_l-\delta\) time, where \(\delta\) is a positive amount of time it is moved backwards. Thus, \(t_a\geq t_l>t_l-\delta\) and so any optimal solution \(S\) can
have its shortest finishing time job \(j_s\) switched into the last position and remain an optimal solution.

Next we consider the set of all jobs minus the shortest completion time job. We order these jobs such that we obtain another optimal solution
for this subset of jobs with a completion time \(t_b\). Our original optimal completion time is \(t_a\), so we know that we can complete this subset of
jobs within time \(t_a\), so \(t_b\leq t_a\). For this subset we can also switch the shortest finishing time job into the last position
without increasing the completion time \(t_b\). Continuing to do this for every subset of decreasing size, we obtain a list of jobs ordered in decreasing
order of finishing times \(f_i\), such that the total completion time is less than or equal to \(t_a\). Thus our optimal solution \(S\) has the same completion
time \(t_a\) as a schedule generated by the shortest finishing time last algorithm, proving that our algorithm generates an optimal solution.

\pagebreak

\section*{Chapter 4, Problem 12}

\paragraph{a)}

This is false. Consider the streams
\[(b_1,t_1)=(10,10),\qquad(b_2,t_2)=(50,1)\]
with \(r=10\). Then \(b_2\nleq rt_2\), but at \(t=10\) we have sent over \(10<rt\) bits, and at \(t=11\) we have sent over \(60<rt\) bits. We have a valid
schedule where \(b_i\nleq rt_i\) is not true for all cases.

\paragraph{b)}

An algorithm to evaluate whether there exists a valid schedule is to check the expression
\[\sum_{i=1}^n b_i \leq r\sum_{i=1}^n t_i\]
If this is true, then a valid schedule exists. If it is false then a valid schedule does not exist. This algorithm runs in \(O(n)\), as it simply
requires a sum over the given streams.

If this expression is false, we prove that a valid schedule does not exist by definition. Our total bits sent is \(\sum_{i=1}^n b_i\), and our finish time
is \(\sum_{i=1}^n t_i\). Thus if \(\sum_{i=1}^n b_i \nleq r\sum_{i=1}^n t_i\), our constraint must be violated by definition at
the end of our schedule because the total number of bits we sent over the time interval from \(0\) to \(\sum_{i=1}^n t_i\) exceeds \(r\sum_{i=1}^n t_i\).

If \(\sum_{i=1}^n b_i \leq r\sum_{i=1}^n t_i\), we can always generate a valid schedule. Let the bitrate of a stream \((b_i,t_i)\) be \(h=\frac{b_i}{t_i}\).
Sending our streams in order of increasing bitrate ensures that we have a valid schedule. We prove this by first defining our average bitrate after \(m\)
streams as
\[h_m=\frac{\sum_{i=1}^m b_i}{\sum_{i=1}^m t_i}\]
We wish to show that \(h_m\leq h_{m+1}\leq h_n\) when we order our bitrates in increasing order, because this means that we never exceed our constraint and
thus we will have a valid schedule since \(h_n\leq r\). We have that
\[h_{m+1}=\frac{\sum_{i=1}^{m+1} b_i}{\sum_{i=1}^{m+1} t_i}=\frac{\sum_{i=1}^m b_i+b_{m+1}}{\sum_{i=1}^m t_i+t_{m+1}}\]
Since we order by increasing bitrate, the stream \(m+1\) has a bitrate \(\frac{b_{m+1}}{t_{m+1}}\) which is greater than any previous stream's bitrate.
Thus, \(\frac{b_{m+1}}{t_{m+1}}\geq h_m\) because \(h_m\) is an average of the previous stream's bitrates. Thus our numerator in \(h_{m+1}\) increases
by a magnitude more than \(h_mt_{m+1}\), so \(h_{m+1}\geq h_m\). Thus we never exceed our constraint and we can generate a valid schedule.

\pagebreak

\section*{Chapter 4, Problem 16}

\pagebreak

\section*{Chapter 4, Problem 30}

\end{document}