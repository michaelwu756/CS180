\documentclass[12pt]{article}
\usepackage{amsmath}
\usepackage{amssymb}
\begin{document}
\title{Computer Science 180, Homework 5}
\date{February 15th, 2018}
\author{Michael Wu\\UID: 404751542}
\maketitle

\section*{Chapter 4, Problem 25}

\pagebreak

\section*{Chapter 4, Problem 28}

Because we must have \(k\) cables from \(X\), we can try every subset of cables of size \(k\) in polynomial time and see if
they can form a spanning tree. Assume there are \(a\) edges from company \(X\). We can generate
\[\binom{a}{k}\]
instances 

\pagebreak

\section*{Chapter 6, Problem 4}

\paragraph{a)}

Let \(M=10\). Then the following operating cost table
\begin{center}
        \begin{tabular}{c | c c}
                & Month 1 & Month 2\\
                \hline
                NY & 1 & 3\\
                SF & 2 & 1\\
        \end{tabular}
\end{center}
results in the given algorithm returning \{NY, SF\} for a cost of \(12\). The correct plan to minimize
cost should be \{SF, SF\} for a cost of \(3\).

\paragraph{b)}

Let \(M=10\). Then the following operating cost table
\begin{center}
        \begin{tabular}{c | c c c c}
                & Month 1 & Month 2 & Month 3 & Month 4\\
                \hline
                NY & 1 & 100 & 1 & 100\\
                SF & 100 & 1 & 100 & 1 \\
        \end{tabular}
\end{center}
results in the only optimal plan \{NY, SF, NY, SF\} for a cost of \(34\). Every optimal plan must move at least
\(3\) times, because the cost of moving is outweighed by the high cost of staying.

\paragraph{c)}

Let \(C(a,b)\) be the cost of operating in city \(a\) during month \(b\). Construct two arrays \(N[i,j]\) and \(S[i,j]\)
that return the cost of operating in New York or San Francisco, respectively, from months \(i\) to \(j\) using the following algorithm.
\begin{verbatim}
for x from 1 to n
    for y from x to n
        if (y==x)
            N[x,y]==C(NY,y)
            S[x,y]==C(SF,y)
        else
            N[x,y]==N[x,y-1]+C(NY,y)
            S[x,y]==S[x,y-1]+C(SF,y)
return N, S
\end{verbatim}
This step runs in \(O(n^2)\) time, as it loops twice over \(n\). Then we define a function \(D(i)\) that gives the minimum operating cost
after \(i\) months. We also define a function \(E(i)\) that gives the end city after executing a minimum operating plan for \(i\) months.
Because there may be multiple ways to plan optimally such that either cities may be the end cities,
\(E(i)\in\{\text{NY},\text{SF},\text{Either}\}\). Let \(M\) be the cost of moving. Then we use the following algorithm to find the
optimal cost after \(n\) months.
\begin{verbatim}
E(0)=Either
D(0)=0
for x from 1 to n
    minCost = infinity
    endCity = Either
    for y from 0 to x-1
        nyMoveCost=0
        sfMoveCost=0
        if (E(y)==NY)
            sfMoveCost=M
        else if (E(y)==SF)
            nyMoveCost=M
        if (D(y)+N(y+1,x)+nyMoveCost<minCost)
            minCost=D(y)+N(y+1,x)+nyMoveCost
            endCity=NY
        if (D(y)+S(y+1,x)+sfMoveCost<minCost)
            minCost=D(y)+S(y+1,x)+sfMoveCost
            endCity=SF
        if ((D(y)+N(y+1,x)+nyMoveCost==minCost && endCity==SF) ||
            (D(y)+S(y+1,x)+sfMoveCost==minCost && endCity==NY))
            endCity=Either
    D(x)=minCost
    E(x)=endCity
return D(n)
\end{verbatim}
This algorithm is \(O(n^2)\) because it loops over \(n\) twice. It is correct because it iteratively generates the next minimum cost for months \(1,\ldots,n\)
using a recurrence relation
\[D(x)=\begin{cases} 0 & x=0\\ \min\limits_{y=0}^{x-1} (D(y)+N(y+1,x)+M_N,D(y)+S(y+1,x)+M_S) & x\geq1 \end{cases}\]
where \(M_N\) and \(M_S\) are equal to \(M\) if the optimal plan \(D(y)\) requires the company to operate in SF or NY, respectively. This is valid because given
optimal plan \(P\) of length \(n\) months, if there are \(m\leq n\) months before the final move, the subset of \(P\) from months \(1\) to \(m\) must be an
optimal plan \(P^\prime\) for the first \(m\) months as well. If \(P^\prime\) was not optimal, we could replace it with a lower cost plan, which is a contradiction
as this means we can reduce the cost of \(P\) by rearranging it. Therefore it makes sense to only calculate the cost after a final move and minimize this value iteratively.
We check if a move is necessary by keeping track of whether or not a minimum cost can be achieved while ending in a particular city using \(E(x)\).

\pagebreak

\section*{Chapter 6, Problem 6}

First, greedily try to fit as many words as possible on each line such that it still remains valid. Let this
greedy formatting have \(x\) lines. Then the sum of slacks will be
\[xL-\left(\sum_{i=1}^n (c_i + 1) - x\right) = x(L+1)-n-\sum_{i=1}^n c_i\]
We wish to minimize the sum of squares of the slacks, which occurs when each slack is approximately the same.
But we also want to keep the same number of lines \(x\), otherwise we add \(L+1\) to the total slack, which is
bad. So we 

\pagebreak

\section*{Chapter 6, Problem 12}

\end{document}